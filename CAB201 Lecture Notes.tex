%!TEX program = xelatex
\documentclass{article}
\usepackage{LaTeX-Submodule/template}

% Additional packages & macros

% Header and footer
\newcommand{\unitName}{Programming Principals}
\newcommand{\unitTime}{Semester 1, 2022}
\newcommand{\unitCoordinator}{Dr Alan Woodley}
\newcommand{\documentAuthors}{\textsc{Tarang Janawalkar}}

\fancyhead[L]{\unitName}
\fancyhead[R]{\leftmark}
\fancyfoot[C]{\thepage}

% Copyright
\usepackage[
    type={CC},
    modifier={by-nc-sa},
    version={4.0},
    imagewidth={5em},
    hyphenation={raggedright}
]{doclicense}

\date{}

\begin{document}
%
\begin{titlepage}
    \vspace*{\fill}
    \begin{center}
        \LARGE{\textbf{\unitName}} \\[0.1in]
        \normalsize{\unitTime} \\[0.2in]
        \normalsize\textit{\unitCoordinator} \\[0.2in]
        \documentAuthors
    \end{center}
    \vspace*{\fill}
    \doclicenseThis
    \thispagestyle{empty}
\end{titlepage}
\newpage
%
\tableofcontents
\newpage
%
% Programming
% Statically typed language
% C# is a statically typed, object oriented language
\section{Types and Expressions}
\begin{definition}[Type]
    The type of an expression is ``what kind of data'' the expression carries.
\end{definition}
\begin{definition}[Variables]
    Variables are a kind of expression which have an \textbf{identity} and a \textbf{value}.
    
    The \textbf{value} of a variable may change as a program runs, however in 
    a statically typed language, the \textbf{type} of each variable is 
    specified before it can be used, and \underline{never changes}.
\end{definition}
\end{document}
